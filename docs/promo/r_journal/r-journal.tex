% !TeX root = RJwrapper.tex
\title{Article Proposal: Concise indicator variable recoding with ind2cat}
\author{by Evangeline Reynolds}

\maketitle

\abstract{%
Indicator variables are easy to create, store, and interpret
\citep{10.1177/1536867X19830921}. They concisely encode information
about the presence or not of a condition for observational units. The
variable name encapsulates the information about the condition of
interest, and the variable's values (TRUE and FALSE, 1 or 0, ``Yes'' or
``No'') indicate if the condition is met for the observational unit.
When using indicator variables to use in summary products, analysts
often make a choice between using an indicator variable as-is or
crafting categorical variables where values can be directly interpreted.
Using the indicator variable as-is may be motivated by time savings, but
yields poor results in summary products. \{\{ind2cat\}\} can help
analysts concisely translate indicator variables to categorical
variables for reporting products, yielding more polished outputs. By
default, ind2cat creates the categorical variable from the indicator
variable name, resulting in a light-weight syntax.
}

\hypertarget{introduction}{%
\section{Introduction}\label{introduction}}

Using current analytic tools, analysts make a choice between directly
using indicator variables or recoding the variable first to categorical.
Current procedures for recoding indicator variables to a categorical
variable is repetitive, but forgoing a recode and using indicator
variables directly yields hard-to-interpret summary products.

The data below inspired by email training data, demonstrates how an
analyst might current recode an indicator variable. This method is
repetitive; in the recoding line, `spam' is typed four times.

\begin{Schunk}
\begin{Sinput}
library(tidyverse)
data.frame(spam = c(TRUE, TRUE, FALSE, FALSE, TRUE)) %>% 
  mutate(cat_spam = ifelse(spam, "spam", "not spam"))
\end{Sinput}
\begin{Soutput}
        spam cat_spam
     1  TRUE     spam
     2  TRUE     spam
     3 FALSE not spam
     4 FALSE not spam
     5  TRUE     spam
\end{Soutput}
\end{Schunk}

Likewise, in data visualization products where recoding can be done on
the fly, we see that the process can be repetative.

\begin{Schunk}
\begin{Sinput}
tidytitanic::passengers %>% 
ggplot() + 
  aes(x = age) + 
  geom_histogram() + 
  facet_grid(~ ifelse(survived, 
                      "survived", 
                      "not survived")) 
\end{Sinput}

\includegraphics[width=0.69\linewidth]{r_journal_files/figure-latex/visual_status_quo-1} \end{Schunk}

Furthermore, the \texttt{ifelse()} approach to recoding indicator
variables also has the disadvantage of not consistently ordering the
resultant categories; ordering in products will be alphabetical and not
reflect the F/T order of the source variable. An additional step to
reflect the source variable, using a function like forcats::fct\_rev,
may be required for consistent reporting. We show this with another
visualization example, and see that the definition of the x axis becomes
easy to reason about.

\begin{Schunk}
\begin{Sinput}
data.frame(ind_grad = c(T, F, T, T)) %>% 
  ggplot() + 
  aes(x = fct_rev(ifelse(ind_grad, "grad", "not grad"))) +
  geom_bar()
\end{Sinput}

\includegraphics[width=0.69\linewidth]{r_journal_files/figure-latex/visual_status_quo_order-1} \end{Schunk}

Given how verbose recoding an indicator variable can be, analysts may
choose to forego a recoding the variable, especially in exploratory
analysis. However, when indicator variables are used directly in data
summary products like tables and visuals, information is often awkwardly
displayed and is sometimes lost. Below, the table that is created by
using the indicator variable directly is awkward to interpret. The
indicator variable name persists in the output allowing savvy readers to
interpret the output, but communication is strained.

\begin{Schunk}
\begin{Sinput}
tidytitanic::passengers %>% 
  count(survived) 
\end{Sinput}
\begin{Soutput}
       survived   n
     1        0 863
     2        1 450
\end{Soutput}
\end{Schunk}

In the following two-way table produced using an indicator variable
directly with the popular janitor package, information is completely
lost:

\begin{Schunk}
\begin{Sinput}
tidytitanic::passengers %>% 
  janitor::tabyl(sex, survived)
\end{Sinput}
\begin{Soutput}
         sex   0   1
      female 154 308
        male 709 142
\end{Soutput}
\end{Schunk}

Likewise, in the following visual summary of the data, where an
indicator variable is directly used, interpretation is awkward.

\begin{Schunk}
\begin{Sinput}
library(tidyverse)

tidytitanic::passengers %>% 
  ggplot() + 
  aes(x = survived) + 
  geom_bar()
\end{Sinput}
\begin{figure}
\includegraphics[width=0.69\linewidth]{r_journal_files/figure-latex/direct_visual_awkward-1} \caption[A]{A. Bar labels + axis label preserves information but is awkward}\label{fig:direct_visual_awkward}
\end{figure}
\end{Schunk}

Moreover, when indicator variables are used directly as faceting
variable for plots produced by the popular ggplot2 library, information
is lost and the plot is not directly interpretable.

\begin{Schunk}
\begin{Sinput}
tidytitanic::passengers %>% 
ggplot() + 
  aes(x = age) + 
  geom_histogram() + 
  facet_grid(~ survived)
\end{Sinput}
\begin{figure}
\includegraphics[width=0.69\linewidth]{r_journal_files/figure-latex/direct_visual_loss-1} \caption[D]{D. Facetting directly on an indicator variable with popular ggplot2 results in information loss}\label{fig:direct_visual_loss}
\end{figure}
\end{Schunk}

\hypertarget{introducing-ind2catind_recode}{%
\section{Introducing
ind2cat::ind\_recode}\label{introducing-ind2catind_recode}}

The ind2cat::ind\_recode() function uses indicator variable names to
automatically derive human-readable, and appropriately ordered
categories.

To clearly compare the new method, we reiterate the status quo with a
toy example:

\begin{Schunk}
\begin{Sinput}
library(tidyverse)

data.frame(ind_graduated = 
             c(TRUE, TRUE, FALSE))  %>% 
  mutate(cat_graduated  = 
           ifelse(ind_graduated, 
                  "graduated", 
                  "not graduated"))  %>% 
  mutate(cat_graduated = 
           fct_rev(cat_graduated)
         )  
\end{Sinput}
\begin{Soutput}
       ind_graduated cat_graduated
     1          TRUE     graduated
     2          TRUE     graduated
     3         FALSE not graduated
\end{Soutput}
\end{Schunk}

Below we contrast this with the use of ind2cat's ind\_recode function
which avoids repetition by creating categories based on the indicator
variable name. Using the the function ind\_recode(), we can accomplish
the same task shown above more succinctly:

\begin{Schunk}
\begin{Sinput}
library(ind2cat)

data.frame(ind_graduated = 
             c(TRUE, TRUE, FALSE)) %>% 
  mutate(cat_graduated  = 
           ind_recode(ind_graduated)
         )
\end{Sinput}
\begin{Soutput}
       ind_graduated cat_graduated
     1          TRUE     graduated
     2          TRUE     graduated
     3         FALSE not graduated
\end{Soutput}
\end{Schunk}

The function ind\_recode is flexible, and can recode from variable
populated with TRUE/FALSE values as well as 1/0 or ``Yes''/``No'' (and
variants `y/n' for example).

Furthermore, while ind\_recode default functionality allows analysts to
move from its first-cut human-readable recode, it also allows fully
customized categories via adjustment of the functions parameters.

If the category associated with `TRUE' should be modified (default is
based on the variable name), the \texttt{cat\_true} may be used as
follows. Note that the false category is generated from the TRUE
category by default.

\begin{Schunk}
\begin{Sinput}
data.frame(ind_graduated = c(T,T,F)) %>% 
  mutate(cat_graduated  = ind_recode(ind_graduated, 
                                     cat_false = "current"))
\end{Sinput}
\begin{Soutput}
       ind_graduated cat_graduated
     1          TRUE     graduated
     2          TRUE     graduated
     3         FALSE       current
\end{Soutput}
\end{Schunk}

Also, the default negator `not' can be changed by setting the
\texttt{negator} argument.

\begin{Schunk}
\begin{Sinput}
tibble(ind_grad = c(T,T,F)) %>%
  mutate(cat_grad  = ind_recode(ind_grad, negator = "~"))
\end{Sinput}
\begin{Soutput}
     # A tibble: 3 x 2
       ind_grad cat_grad
       <lgl>    <fct>   
     1 TRUE     grad    
     2 TRUE     grad    
     3 FALSE    ~ grad
\end{Soutput}
\end{Schunk}

If the negative category should be independently specified, the
\texttt{cat\_false} argument can be set:

\begin{Schunk}
\begin{Sinput}
tibble(ind_grad = c("Y", "N")) %>%
  mutate(cat_grad  = ind_recode(ind_grad, cat_false = "enrolled"))
\end{Sinput}
\begin{Soutput}
     # A tibble: 2 x 2
       ind_grad cat_grad
       <chr>    <fct>   
     1 Y        grad    
     2 N        enrolled
\end{Soutput}
\end{Schunk}

Also, if the derived category's levels should be reversed,
i.e.~{[}1,0{]} instead of the default {[}0,1{]}, rev can be set to TRUE.

\begin{Schunk}
\begin{Sinput}
tibble(ind_grad = c("yes", "no")) %>%
  mutate(cat_grad  = ind_recode(ind_grad, rev = TRUE)) %>% 
  mutate(cat_grad_num = as.numeric(cat_grad))
\end{Sinput}
\begin{Soutput}
     # A tibble: 2 x 3
       ind_grad cat_grad cat_grad_num
       <chr>    <fct>           <dbl>
     1 yes      grad                1
     2 no       not grad            2
\end{Soutput}
\end{Schunk}

Finally, several indicator variable prefixes are automatically removed
with the default setting, includeing \texttt{ind\_} and \texttt{IND\_}.
This behavior can be modified using the \texttt{var\_prefix} argument.

\begin{Schunk}
\begin{Sinput}
tibble(dummy_grad = c(0, 0, 1, 1, 1 ,0 ,0)) %>%
  mutate(cat_grad  = ind_recode(dummy_grad, 
                                var_prefix = "dummy_"))
\end{Sinput}
\begin{Soutput}
     # A tibble: 7 x 2
       dummy_grad cat_grad
            <dbl> <fct>   
     1          0 not grad
     2          0 not grad
     3          1 grad    
     4          1 grad    
     5          1 grad    
     6          0 not grad
     7          0 not grad
\end{Soutput}
\end{Schunk}

\hypertarget{use-in-data-products-like-figures-and-tables}{%
\subsection{Use in data products like figures and
tables}\label{use-in-data-products-like-figures-and-tables}}

In the summary figure, we show the values that result from using
ind\_recode on the fly in ggplot2. In a true-to-life analytic reporting
space, the analyst could then use \texttt{labs(x\ =\ NULL)} to remove
the variable recoding specification.

\begin{Schunk}
\begin{Sinput}
data.frame(ind_spam = c(TRUE, TRUE, FALSE, FALSE, FALSE)) %>% 
ggplot() + 
  aes(x = ind_recode(ind_spam)) + 
  geom_bar() +
  theme_gray(base_size = 15)->
p1

p1 +
  aes(x = ind_recode(ind_spam, cat_true = "suspicious")) ->
p2

p1 +
  aes(x = ind_recode(ind_spam, negator = "~")) ->
p3

p1 +
  aes(x = ind_recode(ind_spam, cat_false = "trustworthy")) ->
p4


p1 +
  aes(x = ind_recode(ind_spam, rev = TRUE)) ->
p5

library(patchwork)

(p1 + p2) /
  (p3 + p4) /
  (p5 + patchwork::plot_spacer())
\end{Sinput}

\includegraphics[width=0.69\linewidth]{r_journal_files/figure-latex/visual_ind2cat_customization_in_visualizations-1} \end{Schunk}

\begin{Schunk}
\begin{Sinput}
tidytitanic::passengers %>%
  mutate(cat_survived = ind_recode(survived, 
                                   cat_false = "perished")) %>% 
  janitor::tabyl(sex, cat_survived) %>% 
  janitor::adorn_percentages() %>% 
  janitor::adorn_pct_formatting() %>% 
  janitor::adorn_ns(position = "rear")
\end{Sinput}
\begin{Soutput}
         sex    perished    survived
      female 33.3% (154) 66.7% (308)
        male 83.3% (709) 16.7% (142)
\end{Soutput}
\end{Schunk}

\hypertarget{conclusion}{%
\section{Conclusion}\label{conclusion}}

\hypertarget{implementation-details}{%
\section{Implementation details}\label{implementation-details}}

\begin{Schunk}
\begin{Sinput}
readLines("R/ind_recode.R") -> implementation
\end{Sinput}
\end{Schunk}

\begin{Schunk}
\begin{Sinput}
#' ind_recode
#'
#' @param var the name of an indicator variable
#' @param var_prefix a character string that will be ignored when creating the categorical variable
#' @param negator a character string used to create cat_false when cat_false is NULL, default is 'not'
#' @param cat_true a character string string to be used place of  T/1/"Yes" for the categorical variable output, if NULL the category is automatically generated from the variable name
#' @param cat_false a character string string to be used place of  F/0/"No" for the categorical variable output, if NULL the category is automatically generated from the cat true and the negator
#' @param rev logical indicating if the order should be reversed from the F/T ordering of the indicator source variable, default is FALSE
#'
#' @return
#' @export
#'
#' @examples
#' library(tibble)
#' library(dplyr)
#' tibble(ind_grad = c(0,0,1,1,1 ,0 ,0)) %>%
#'   mutate(cat_grad  = ind_recode(ind_grad))
#'
#' tibble(ind_grad = c(TRUE,TRUE,FALSE)) %>%
#'   mutate(cat_grad  = ind_recode(ind_grad))
#'
#' tibble(ind_grad = c("Y", "N")) %>%
#'   mutate(cat_grad  = ind_recode(ind_grad))
#'
#' tibble(ind_grad = c("y", "n")) %>%
#'   mutate(cat_grad  = ind_recode(ind_grad))
#'
#' tibble(ind_grad = c("yes", "no")) %>%
#'   mutate(cat_grad  = ind_recode(ind_grad))
ind_recode <- function(var, var_prefix = "ind_", negator = "not",
                       cat_true = NULL, cat_false = NULL, rev = FALSE){

  if(is.null(cat_true)){
    cat_true = deparse(substitute(var)) %>%   # use r lang in rewrite
      stringr::str_remove(paste0("^", var_prefix)) %>%
      stringr::str_replace_all("_", " ")
  }

  if(is.null(cat_false)){
    cat_false = paste(negator, cat_true)
  }

  # for yes/no case - dangerously.
  if(is.character({{var}})){

    my_var <- {{var}} %>% as.factor() %>% as.numeric() - 1

  }else{

    my_var <- {{var}}
  }

  if(rev){
    ifelse(my_var, cat_true, cat_false) %>%
      factor(levels = c(cat_true, cat_false))
  }else{
    ifelse(my_var, cat_true, cat_false) %>%
      factor(levels = c(cat_false, cat_true))
  }


}
\end{Sinput}
\end{Schunk}

\begin{center}\rule{0.5\linewidth}{0.5pt}\end{center}

\hypertarget{readme.rmd-chunks-names}{%
\section{README.Rmd chunks names}\label{readme.rmd-chunks-names}}

\begin{Schunk}
\begin{Sinput}
knitr::knit_code$get() |> names()
\end{Sinput}
\begin{Soutput}
      [1] "unnamed-chunk-1"                               
      [2] "manipulation_status_quo"                       
      [3] "visual_status_quo"                             
      [4] "visual_status_quo_order"                       
      [5] "direct_table_awkward"                          
      [6] "direct_table_loss"                             
      [7] "direct_visual_awkward"                         
      [8] "direct_visual_loss"                            
      [9] "manipulation_status_quo_reprise"               
     [10] "manipulation_ind2cat"                          
     [11] "manipulation_ind2cat_custom"                   
     [12] "manipulation_ind2cat_negator"                  
     [13] "manipulation_ind2cat_false_cat"                
     [14] "manipulation_ind2cat_rev"                      
     [15] "manipulation_ind2cat_prefix"                   
     [16] "visual_ind2cat_customization_in_visualizations"
     [17] "table_ind2cat_preserves"                       
     [18] "unnamed-chunk-3"                               
     [19] "unnamed-chunk-4"                               
     [20] "unnamed-chunk-5"
\end{Soutput}
\end{Schunk}

\bibliography{RJreferences.bib}

\address{%
Evangeline Reynolds\\
Affiliation\\%
line 1\\ line 2\\
%
\url{https://journal.r-project.org}\\%
\textit{ORCiD: \href{https://orcid.org/0000-0002-9079-593X}{0000-0002-9079-593X}}\\%
\href{mailto:author1@work}{\nolinkurl{author1@work}}%
}

